\setlength{\parindent}{2em} % 首行缩进两个字符

% 行距
\renewcommand{\normalsize}{\fontsize{11pt}{15pt}\selectfont}

% 西文 Serif
\setmainfont{lmroman10-regular}[
	Path = ./fonts/LatinModern/,
	Extension = .otf,
	BoldFont = lmroman10-bold,
	ItalicFont = lmroman10-italic,
	BoldItalicFont = lmroman10-bolditalic
]

% 中文 Serif
\setCJKmainfont{SourceHanSerifSC-Regular}[
	Path = ./fonts/SourceHanSerifSC/,
	Extension = .otf,
	BoldFont = SourceHanSerifSC-Bold,
	ItalicFont = SourceHanSerifSC-Light,
	UprightFont = SourceHanSerifSC-Regular,
	MediumFont = SourceHanSerifSC-Medium,
	HeavyFont = SourceHanSerifSC-Heavy,
	ExtraLightFont = SourceHanSerifSC-ExtraLight,
	SemiBoldFont = SourceHanSerifSC-SemiBold
]

% 中文 Sans
\setCJKsansfont{SourceHanSansSC-Regular}[
	Path = ./fonts/SourceHanSansSC/,
	Extension = .otf,
	BoldFont = SourceHanSansSC-Bold,
	LightFont = SourceHanSansSC-Light,
	UprightFont = SourceHanSansSC-Regular,
	MediumFont = SourceHanSansSC-Medium,
	HeavyFont = SourceHanSansSC-Heavy,
	ExtraLightFont = SourceHanSansSC-ExtraLight,
	NormalFont = SourceHanSansSC-Normal
]

% 数学字体
\setmathfont{LMRomanMath}[
	Path = ./fonts/LMRomanMath/,
	Extension = .otf
]

% Nerd 符号图标字体
\newfontfamily\nerd{SymbolsNerdFont-Regular}[
	Path = ./fonts/Nerd/,
	Extension = .ttf
]

% Avali Scratch 字体
\newfontfamily\avaliscratch{avali-scratch}[
	Path = ./fonts/Avali_Scratch/,
	Extension = .otf
]
