\ChapterTitle{在冰雪中的飞羽}

一颗气态巨行星紧挨着一颗不起眼的冰卫星,这颗卫星叫 Avalon,厚重的云层将这颗冰冷的岩石卫星与外界隔绝。
随着这颗卫星的转动,栖息于此的生物依然过着它们的日子,对云层外的世界一无所知,因为这里早已是它们世世代代以来的家园。

咔哒敲击声和羽毛的轻微颤动声清晰可闻,只有风声和摇曳的异星植物能将其削弱一部分。
生物在冰雪上奔跑,身后是由钢铁和塑料组成的机器。科技在世界各地安静地运转着,并由一群自称 Avali 的冰猛禽精心维护和操控。

Avali 一如既往地度过它们的一天,严寒对它们来说早已习以为常。Avali 穿行于寒风和厚厚的积雪,仿佛这些阻碍根本不存在。
它们生活在这里,在这里演化,对它们来说,这就像是在海滩上享受温暖的一天。

% ! 此部分对`The Avali are small, feathered creatures that evolved to have complex problem-solving skills.'进行了裁剪,删去了`演化出了复杂的问题解决能力`。
% ! 因为这句话会使读者迷惑,大多数读者并不清楚为什么需要靠演化得到解决复杂问题的能力,而进行修改又会严重偏离原意思,故删除。
Avali 是一种长有羽毛的小型生物,长得很像猛禽,有着两对耳朵用来辨别方位。虽然长有眼睛,但耳朵才是主要的感官。这些生物经常成群出现。
可爱的外表让大多数旁观者觉得它们毫无威胁。但它们那锋利的爪子和牙齿不可小觑。
它们还开发了进入太空的技术,这使得它们可以从其恒星系统内获取资源。

\subsection*{起源}

Avali(复数“Avali”,形容词“Avalian”)其概念由 RyuujinZERO 创造。
最初该种族叫“Iubati”,源自拉丁语,意思是“冠(crest)”或“有冠的(crested)”,在 EA-Maxis 的游戏“孢子(Spore)”中被创建,于 2009 年初首次亮相,并发布在 “Sporepedia” 上。
随后,Ryuujin 开始为它们构建设定基础,其中包括冰冻的家园(母星)、大量使用的技术和人工智能算法,以及 Iubati 实际采用的一些载具概念。

冒险故事是使用 EA-Maxis 的“孢子:银河冒险”扩展包构建的。
后来,Ryuujin 在《孢子》中制作了一个叫“Whom Gods Destroy”的自定义故事任务。
% ! 此处的`artifacts`翻译暂定`遗物`
这个故事融合了科幻、恐怖与悲剧元素,将玩家置于船长的视角,在贫瘠的世界中寻找遗物,结果他们的小冒险却出了差错。

2014 年,Ryuujin 开始为呵呵鱼(Chucklefish)即将推出的游戏“星界边境(Starbound)”开发新种族。她采用了“孢子”中的 Avali 概念并继续发展完善该物种。
在《星界边境》中,Avali 的社会和群居系统、饮食习惯、文化和生物学特征等都得到了进一步的完善。
在2014 年 4 月,Ryuujin 正式发布了《星界边境》的 Avali 模组。

\clearpage

% TODO:`A Galaxy Awaits`需重译
\subsection*{一个星系在等待着}

\AddToShipoutPictureBG*{
	\put(\dimexpr\paperwidth-0.6\paperwidth-1cm\relax,-1cm){
		\includegraphics[width=0.7\paperwidth]{chapters/chapter1/ICE_AND_FEATHER/assets/A_Galaxy_Awaits_Background.png}
	}
}

如今,Avali 的生命力仍在粉丝群体中延续。
这得益于它们可爱的外表和在网络上的高活跃度,这些小型冰猛禽已经形成了一个活跃的社群。
许多游戏模组和同人作品将 Avali 融入到新的故事和冒险中,粉丝们也持续创作着相关的同人画作和故事。

自2015年达到巅峰以来,Avali 种族的官方发展就陷入了停滞。
虽然 Ryuujin 并未放弃这个种族,但她对 Avali 的重视程度已不如《星界边境》初期开发阶段。
作为回应,玩家社区逐渐接手并继续发展这个种族,为 Avali 创造了不同的背景故事和文化。
这种开放性使得 Avali 能够融入原本难以融入的宇宙设定中。

本书虽是对 Avali 种族的全面指南,但并非官方设定。
Avali 种族的设定在本书中被汇编成章节和小节。
每一部分都将从主观视角出发,细致入微地剖析和解释 Avali 种族的某些方面。
此外,本书还将沿用 Ryuujin 的故事,继续讲述 Avali 种族的发展历程。

\clearpage
